\documentclass{article}

\usepackage{fancyhdr}
\usepackage{extramarks}
\usepackage{amsmath}
\usepackage{amsthm}
\usepackage{amsfonts}
\usepackage{tikz}
\usepackage[plain]{algorithm}
\usepackage{algpseudocode}
\usepackage{enumerate}
\usepackage{graphicx}
\usepackage{pythonhighlight}
\usepackage{amssymb}

\usetikzlibrary{automata,positioning}

%
% Basic Document Settings
%  

\topmargin=-0.45in
\evensidemargin=0in
\oddsidemargin=0in
\textwidth=6.5in
\textheight=9.0in
\headsep=0.25in

\linespread{1.1}

\pagestyle{fancy}
\lhead{\hmwkAuthorName}
\chead{\hmwkClass\ (\hmwkClassInstructor): \hmwkTitle}
\rhead{\firstxmark}
\lfoot{\lastxmark}
\cfoot{\thepage}

\renewcommand\headrulewidth{0.4pt}
\renewcommand\footrulewidth{0.4pt}

\setlength\parindent{0pt}

%
% Create Problem Sections
%

\newcommand{\enterProblemHeader}[1]{
    \nobreak\extramarks{}{Problem \arabic{#1} continued on next page\ldots}\nobreak{}
    \nobreak\extramarks{Problem \arabic{#1} (continued)}{Problem \arabic{#1} continued on next page\ldots}\nobreak{}
}

\newcommand{\exitProblemHeader}[1]{
    \nobreak\extramarks{Problem \arabic{#1} (continued)}{Problem \arabic{#1} continued on next page\ldots}\nobreak{}
    \stepcounter{#1}
    \nobreak\extramarks{Problem \arabic{#1}}{}\nobreak{}
}

\setcounter{secnumdepth}{0}
\newcounter{partCounter}
\newcounter{homeworkProblemCounter}
\setcounter{homeworkProblemCounter}{1}
\nobreak\extramarks{Problem \arabic{homeworkProblemCounter}}{}\nobreak{}

%
% Homework Problem Environment
%
% This environment takes an optional argument. When given, it will adjust the
% problem counter. This is useful for when the problems given for your
% assignment aren't sequential. See the last 3 problems of this template for an
% example.
%
\newenvironment{homeworkProblem}[1][-1]{
    \ifnum#1>0
        \setcounter{homeworkProblemCounter}{#1}
    \fi
    \section{Problem \arabic{homeworkProblemCounter}}
    \setcounter{partCounter}{1}
    \enterProblemHeader{homeworkProblemCounter}
}{
    \exitProblemHeader{homeworkProblemCounter}
}

%
% Homework Details
%   - Title
%   - Due date
%   - Class
%   - Section/Time
%   - Instructor
%   - Author
%

\newcommand{\hmwkTitle}{Homework\ \#2}
\newcommand{\hmwkDueDate}{November 24, 2020}
\newcommand{\hmwkClass}{SI114H}
\newcommand{\hmwkClassInstructor}{Professor Qifeng Liao}
\newcommand{\hmwkAuthorName}{Tianyuan Wu}
\newcommand{\hmwkAuthorID}{63305667}

%
% Title Page
%

\title{
    \vspace{2in}
    \textmd{\textbf{\hmwkClass:\ \hmwkTitle}}\\
    \normalsize\vspace{0.1in}\small{Due\ on\ \hmwkDueDate\ at 11:59pm}\\
    \vspace{0.1in}\large{\textit{\hmwkClassInstructor}}
    \vspace{3in}
}

\author{\textbf{\hmwkAuthorName}\\ \hmwkAuthorID}
\date{}

\renewcommand{\part}[1]{\textbf{\large Part \Alph{partCounter}}\stepcounter{partCounter}\\}

%
% Various Helper Commands
%

% Useful for algorithms
\newcommand{\alg}[1]{\textsc{\bfseries \footnotesize #1}}

% For derivatives
\newcommand{\deriv}[1]{\frac{\mathrm{d}}{\mathrm{d}x} (#1)}

% For partial derivatives
\newcommand{\pderiv}[2]{\frac{\partial}{\partial #1} (#2)}

% Integral dx
\newcommand{\dx}{\mathrm{d}x}

% Alias for the Solution section header
\newcommand{\solution}{\textbf{\large Solution}}

% Probability commands: Expectation, Variance, Covariance, Bias
\newcommand{\E}{\mathrm{E}}
\newcommand{\Var}{\mathrm{Var}}
\newcommand{\Cov}{\mathrm{Cov}}
\newcommand{\Bias}{\mathrm{Bias}}

\begin{document}

\maketitle

\pagebreak

\begin{homeworkProblem}
    \solution\\
    \begin{align}\nonumber
        \hat{f}(k) &= \int_{-\infty}^{\infty}f(x)e^{-ikx}dx\\
        &= \int_{0}^{\infty}e^{-(ik+a)x}dx + \int_{-\infty}^{0}e^{(a-ik)x}dx \\
        &= \frac{1}{a+ik} - \frac{1}{a-ik}\\
        &= \frac{-2ik}{a^2 + k^2}
    \end{align}
    So, the decay rate of $\hat{f}(k)$ is $\frac{1}{k^2}$, and there is a jump discontinuity in $g(x)$.
\end{homeworkProblem}

\begin{homeworkProblem}
    \begin{enumerate}
    \item [(a)]
    \begin{align}\nonumber
        \hat{f}(k) &= \int_{0}^{L} 1\cdot e^{-ikx}dx\\
        &= -\frac{1}{ik}(e^{-ikL} - 1)\\
        &= \frac{1}{ik}
    \end{align}

    \item [(b)]
    We can calculate this fourier transformation by calculate the limit of $a=0$ of problem1
    \begin{align}\nonumber
        \hat{f}(k) &= \lim_{a\rightarrow 0}\frac{-2ik}{a^2+k^2}\\
        &= \frac{-2i}{k}
    \end{align}

    \item [(c)]
    Notice that $f(x) = \int_{0}^{1}e^{ikx}dx$ is the inverse fourier transform of the following function:
    \begin{align}\nonumber
        g(x) = \begin{cases}
            1, \quad 0\le x \le 1\\
            0, \quad \text{otherwise}
        \end{cases}
    \end{align}
    Hence, the fourier transform of $f(x)$ is
    \begin{align}\nonumber
        \hat{f}(k) = \begin{cases}
            1, \quad 0\le k \le 1\\
            0, \quad \text{otherwise}
        \end{cases}
    \end{align}

    \item [(d)]
    \begin{align}\nonumber
        \hat{f}(k) &= \int_{0}^{4\pi}sin(x)e^{-ikx}dx\\
        &= \int_{0}^{4\pi}\frac{e^{ix} - e^{-ix}}{2i}e^{-ikx}dx\\
        &= \frac{1}{2i}\frac{1}{(1-k)i}(e^{(1-k)4\pi i} - 1) - \frac{1}{2i}\frac{1}{(1+k)i}(1 - e^{-(1+k)4\pi i})\\
        &= \frac{1}{1-k^2}
    \end{align}
    \end{enumerate}
\end{homeworkProblem}

\begin{homeworkProblem}
    \begin{enumerate}
        \item [(a)]
        By the property of delta function $\delta(x)$, we know
        \begin{align}\nonumber
            f(x) &= \int_{-\infty}^{\infty}\delta(k)e^{ikx}dk\\
            &= 1
        \end{align}
        
        \item [(b)]
        \begin{align}\nonumber
            f(x) &= \int_{-\infty}^{0}e^ke^{ikx}dk + \int_{0}^{\infty}e^{-k}e^{ikx}dk\\
            &= \lim_{w\rightarrow 0}(\int_{-w}^{0}e^ke^{ikx}dk + \int_{0}^{w}e^{-k}e^{ikx}dk)\\
            &= \lim_{w\rightarrow 0}(\frac{2}{x^2+1} + e^{-w}[ix(e^{-ixw} - e^{ixw}) - (e^{-ixw} + e^{ixw})])\\
            &= \frac{2}{x^2+1}
        \end{align}
        \end{enumerate}
\end{homeworkProblem}

\end{document}
